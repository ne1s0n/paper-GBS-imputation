\section{Materials and methods}

\subsection{Plant material}
\label{sec:plant_material}

Samples from two autotetraploid alfalfa (\emph{Medicago sativa L.}) and one diploid rice (\emph{Oryza sativa L.}) populations were available (see Table~\ref{tab:descriptive_statistics}).\\
Alfalfa populations included elite germplasms from the Po Valley (Alfalfa-PV) and Mediterranean-climate environments (Alfalfa-Med). For Alfalfa-PV, 124 parent genotypes were chosen by stratified mass selection for dry matter yield over three harvests. For Alfalfa-Med, 154 parent genotypes were derived from two cycles of free intercrossing among three outstanding populations in a multi-environment study. For further details see \cite{Annicchiarico2015}. \\
The Rice dataset included 437 plants belonging to 391 accessions (46 duplicates) from the Rice Germplasm Collection maintained at CRA-Rice Research Unit (Vercelli, Italy). The sampled collection comprised accessions from the five main sub-populations of \emph{O. sativa} (274 temperate japonica, 108 tropical japonica, 28 indica, 16 aus and 11 aromatic). All accessions were purified through single seed descent, and were genotypically essentially 100\% homozygous.

\subsection{Genotyping by sequencing}
\label{sec:overview}

The 715 samples from alfalfa (Alfalfa-Med and Alfalfa-PV) and rice (Rice) populations were genotyped using the genotyping-by-sequencing (GBS) approach \cite{elshire_robust_2011}. Different GBS protocols were used to genotype the alfalfa and rice samples, since genotyping was carried out under different projects.\\
In alfalfa, DNA was isolated from fresh leaf tissues by the Wizard~\textregistered~Genomic DNA Purification Kit (Promega, A1125) and quantified with a Quant-iT PicoGreen dsDNA assay kit (Life Technologies, P7589). Two libraries were constructed for the two populations, where 100 ng of each DNA was digested with ApeKI (NEB, R0643L) and then ligated to a unique barcoded adapter and a common adapter. For the reference population Alfalfa-Med, 5 nmoles each of the primers and NEB 2X Taq Master Mix (NEB Cat \# M0270S) were included in the PCR reaction according to \cite{elshire_robust_2011} original protocol, while for Alfalfa-PV the KAPA library amplification readymix (Kapa Biosystems Cat \# KK2611) was used instead. Each library was sequenced in two lanes on Illumina HiSeq 2000 at the Genomic Sequencing and Analysis Facility at the University of Texas at Austin, Texas, USA. For both populations post sequencing analysis and SNP calling was carried out using Tassel UNEAK pipeline \cite{lu_switchgrass_2013}.

Alfalfa is an autotetraploid plant species and can therefore present three different heterozygous genotypes: AAAB, AABB and ABBB. Sequencing depth in this study was not sufficient for accurate tetraploid allelic dosage, but following \cite{li_saturated_2014} and \cite{li_genomic_2015} reliable genotype calls based on diploid allelic dosage were obtained considering diploid heterozygotes (i.e. AB), while the two homozygous genotypes (AAAA and BBBB) were considered diploid homozygotes (i.e., AA or BB). A further quality filter, implemented through ad-hoc Python scripts, removed heterozygous loci with less than 4 and homozygous loci with less than 11 aligned reads. A similar filtering was performed in \cite{Rocher_validation_2015} using less restrictive thresholds. 

In rice, DNA was isolated from three-weeks old leaves using the DNeasy Plant Mini Kit (QIAGEN) with a TECAN Freedom EVO150 liquid handling robot (TECAN Group Ltd, Switzerland). DNA digestion was performed on 100 ng samples in 96-well plates using ApeKI, which was shown to cut every 1 kb on average in a in-silico digestion of the Nipponbare reference genome, and 96-plex libraries constructed according to the GBS protocol. The libraries were loaded into Genome Analyzer II (Illumina, Inc., San Diego, CA) for sequencing. Raw sequence data filtering, sequence alignment to the rice reference genome (Os-Nipponbare-Reference-IRGSP-1.0, \cite{kawahara_improvement_2013}), and SNP calling from GBS genotyping were carried out using the Tassel GBS pipeline \cite{glaubitz_tassel-gbs:_2014}.\\
In total, 32\,706, 40\,734 and 166\,418 SNP markers were called by GBS in Alfalfa-PV, Alfalfa-Med and Rice, respectively (see Table~\ref{tab:descriptive_statistics}). In both datasets SNP variants were renamed so that AA represented the most common homozygote, and BB the least common homozygote.

\begin{table}
\centering
\caption[Descriptive statistics]{
Descriptive statistics of alfalfa (\emph{Medicago sativa L.}) 
and rice (\emph{Oryza sativa L.}) genotyping. MAF is average minor allele frequency; 
p(AA), p(AB) and p(BB) are the proportions of AA, AB and BB genotypes.}
\label{tab:descriptive_statistics}
\begin{tabular}{lccccccc}
\hline\noalign{\smallskip}
\noalign{\smallskip}\hline\noalign{\smallskip}
\textbf{Dataset} & \textbf{No. samples} & \textbf{No. SNPs} & \textbf{Missing Rate} & \textbf{MAF} & \textbf{p(AA)} & \textbf{p(AB)} & \textbf{p(BB)}\\
\noalign{\smallskip}\Xhline{3\arrayrulewidth}\noalign{\smallskip}
Alfalfa-PV & 124 & 32\,706 & 0.666 & 0.1724 & 0.690 & 0.274 & 0.035\\
Alfalfa-Med & 154 & 40\,734 & 0.596 & 0.1702 & 0.691 & 0.276 & 0.032\\
Rice & 437 & 166\,418 & 0.534 & 0.140 & 0.860 & - & 0.140 \\
\noalign{\smallskip}\hline
\end{tabular}
\end{table}


\subsection{Imputation methods}
\label{sec:imputation_methods}
We considered six imputation methods: Mean Value Imputation (MNI), K-Nearest Neighbour Imputation (KNNI), Random Forest Imputation (RFI), Singular Value Decomposition Imputation (SVDI), Localized Haplotype Clustering Imputation (``Beagle'') and \textcolor{red}{haplotype reconstruction around the recombination site}
(``FILLIN'' from Tassel framework). For all algorithms we imputed a $m\times n$ matrix of m individuals and n markers whose data points, defined in the set \{0,1,2,NA\}, represented the three possible genotypes (AA, AB, and BB) and the missing value, respectively. 
\paragraph{MNI}
\label{par:MNI}
: in Mean Value Imputation each missing data point was replaced by the mean of the non-missing values for that marker, then discretized to the closest value in \{0,1,2\}.
\paragraph{KNNI}
\label{par:KNNI}
: in K-Nearest Neighbors Imputation missing data points were imputed based on the weighted average of the K closest markers \cite{troyanskaya_missing_2001} defined by the Simple Matching Coefficient distance function \cite{schwender_statistical_2007}, specifically designed for categorical data. $K=4$ was used in KNNI, and neighbour values were weighted by the reciprocal of their distance from the data point to be imputed.
\paragraph{SVDI}
\label{par:SVDI}
: Singular Value Decomposition (SVD) imputation was based on the following factorization of the genotype matrix M (m markers, n individuals):

\begin{equation}
\label{eq:SVDI_general}
M = U\Sigma V^{T}
\end{equation}

where U is a $m \times m$ unitary matrix (i.e. $UU^{T}=I$), $\Sigma$ is a $m \times n$ rectangular diagonal matrix of singular values and $V^{T}$ is the $n \times n$ conjugate transpose of the unitary matrix V. Matrix U and V contain the eigenvectors of $M^{T}M$ and $MM^{T}$, respectively, and the corresponding non-zero singular values in $\Sigma$ are equivalent to the square-root of the eigenvalues of $M^{T}M$ and $MM^{T}$. The first k eigenvectors - ordered by decreasing eigenvalue - capture most of the information in the marker genotype matrix, and were used to generate linear combinations (principal components) of the original m markers for the imputation of missing data points. The imputation procedure comprised: (i) initial imputation of missing genotypes using MNI, since SVD can only be performed on complete matrices; (ii) SVD to select the most informative k eigenvectors of the marker genotype matrix; (iii) these k eigenvectors were used as predictors in a linear regression model for marker genotypes:

\begin{equation}
\label{eq:SVDI_regression}
Y = U^*\beta +\varepsilon 
\end{equation}

with Y as vector of genotypes at marker j; $U^*$ matrix of the first k eigenvectors; $\beta$ vector of regression coefficients; $\varepsilon$ random errors terms; (iv) all eigenvectors (matrix U) and $\beta$ were used to estimate missing values at marker j. The procedure was repeated (steps (ii) to (iv)) until convergence. The final imputed data points were then discretized to the closer genotype in \{0,1,2\}. Additional details on SVDI can be found in \cite{troyanskaya_missing_2001}

\paragraph{RFI}
\label{par:RFI}
: in Random Forest (RF) imputation, missing genotypes at marker j were imputed by means of RF multiple regression trees \cite{breiman_random_2001} where all markers other than j were used for the prediction. At each marker j, 100 RF regression trees $\Theta_{1} \ldots \Theta_{100}$ were grown from a bootstrapped sample of the individuals in Y and a random subset of $\sqrt{m-1}$ markers. Missing genotypes were imputed averaging predictions over the 100 RF trees

\begin{equation}
\label{eq:RFI_regression}
\hat{Y} = \frac{1}{100}\sum_{i=1}^{100} h(x,\Theta_i)
\end{equation}

where $h(x,\Theta_i)$ is a function of the $i_{th}$ tree and subset of markers. RFI was repeated until convergence or for maximum 10 iterations. After regression, the imputed data were then discretized to the closer genotype in \{0,1,2\}. 

\paragraph{Beagle}
\label{par:Beagle}
: ``localized haplotype clustering imputation'' is a method implemented in the software ``Beagle'' \cite{browning_rapid_2007}. Originally developed for human genetics, Beagle has since found wide application also in animal and plant genetics. Beagle has become so popular that the name of the software is commonly used to refer metonymically to the method it implements, making the two hardly distinguishable. Beagle infers haplotypes and imputes sporadic missing alleles both with known and unknown phase, using a localized haplotype cluster model. This is a class of directed acyclic graphs which empirically models haplotype frequencies on a local scale and therefore adapts to local structure in the data. Beagle makes use of a hidden Markov model to find the most likely haplotype pair for each individual, given the genotype data for that individual and the graphical haplotype frequency model. The method works iteratively using an expectation-maximization (EM) approach. The imputed missing data, probabilities of missing genotypes and inferred haplotypes are calculated from the model that is fitted in the last iteration. For the imputation of missing genotypes in the completely homozygous rice accessions, a likelihood file (with the prior likelihood of each of the three possible SNP genotypes) was defined to specifically exclude the imputation of non-possible heterozygous genotypes (details in \cite{beagle3_manual}).

\paragraph{FILLIN}
The ``Fast Inbred Line Library ImputatioN'' (``FILLIN'') is an imputation method optimised for inbred populations implemented in the ``Tassel'' programming suite \cite{swarts_novel_2014}. FILLIN is based on haplotype reconstruction around recombination break points. Haplotypes are clustered per genotype similarity using the Hamming distance function. This information is eventually used to impute the target locus in an iterative approach that attemps, through a Markov Chain MonteCarlo (MCMC) process, to maximise the likelihood of the observed SNP calls given the unobserved imputed genotypes. If convergence criteria are not met genotypes are left uninputed. Among the considered algorithms, FILLIN is the only one that can thus produce partially imputed results.

\subsection{Imputation procedure}
\label{sec:imputation_procedure}
In order to assess the imputation performance of the different methods, we introduced increasing proportions of artificial missing genotypes in the data sets. These were then imputed with the various algorithms, measuring the resulting imputation accuracy and computation time. We extracted four datasets from the overall data that had a maximum of 10\%, 20\%, 40\% or 70\% missing data per marker. For each of these datasets, we randomly introduced an additional 1\%, 5\%, 10\% or 20\% missing genotypes, on which imputation accuracy could be measured. Table~\ref{tab:missing_rates} reports the number of markers and average missing rate (before introducing artificial missing genotypes) for the four call-rate thresholds. This procedure produced 16 combinations for each population: 4 call rate datasets $\times$ 4 levels of artificial missing markers. On each combination, we applied the 6 described imputation methods. Rice data were analysed as a whole and by each of the 12 chromosomes separately. In absence of alfalfa genome, the \emph{M. Truncatula} genome was used as reference for those algorithms (Beagle and FILLIN) requiring markers positions. 

To further investigate the effect of presence (or absence) of genome information we generated ``reshuffled'' datasets where marker positions were randomly assigned inside each chromosome. This way, the linkage disequilibrium between markers is broken and the relative performance decrease can be directly assessed. The reshuffled datasets were tested only on Beagle and FILLIN.  


\begin{table}
\centering
\caption[Markers counts and missing rates]{
Total number of markers and resulting average missing rate for the four different allowed missing rate thresholds in alfalfa (two datasets) and rice.}
\label{tab:missing_rates}
\begin{tabular}{C{3cm}rccc}
\hline\noalign{\smallskip}
\noalign{\smallskip}\hline\noalign{\smallskip}
\textbf{Allowed missing rate threshold} &  & \textbf{Alfalfa-PV} & \textbf{Alfalfa-Med} & \textbf{Rice}\\
\noalign{\smallskip}\Xhline{3\arrayrulewidth}\noalign{\smallskip}
   \multirow{2}{*}{70\%} & Missing rate &  0.325 & 0.29  & 0.364\\
                           & Markers      & 13\,190  & 19\,986 & 109\,372\\    
\noalign{\smallskip}\hline
   \multirow{2}{*}{40\%} & Missing rate &  0.161 & 0.142 & 0.198\\
                           & Markers      & 7\,790 & 12\,931 & 58\,553\\    
\noalign{\smallskip}\hline
   \multirow{2}{*}{20\%} & Missing rate & 0.076 & 0.074 & 0.099\\
                           & Markers      & 4\,828 & 8\,962 & 29\,872\\
\noalign{\smallskip}\hline                           
   \multirow{2}{*}{10\%} & Missing rate & 0.046 & 0.045 & 0.049\\
                           & Markers      & 3\,405 & 6\,364 & 15\,060 \\
\noalign{\smallskip}\hline
\end{tabular}
\end{table}









\subsection{Imputation accuracy and efficiency assessment}
\label{sec:imputation_accuracy_and_efficiency_assessment}
We used the artificial missing genotypes to measure the performance of the six imputation methods. The rice dataset contained only two genotype classes: AA (homozygous for the major allele) and BB (homozygous for the minor allele). Both alfalfa datasets contained also the third genotype classes AB (heterozygous, pooling the three possible heterozygotes AAAB, AABB, ABBB).\\
For each experiment we measured the overall imputation accuracy and the imputation accuracy within each genotype class. The imputation accuracy was computed as the number of missing data correctly imputed divided by the total number of artificially missing data (proportion of correct imputations):

\begin{equation}
\label{eq:accuracy}
accuracy = \frac{1}{n} \sum_{i=1}^{n}{I(g_i = \hat{g_i})}
\end{equation}

where $I()$ is an indicator function that returns 1 if the imputed ($\hat{g}$) and true ($g$) genotypes are equal, 0 otherwise. We obtained four accuracy measures for alfalfa and three for rice.\\
For each imputation method, the total computation time needed to complete the analyses was measured as an indicator of their relative performance. The measured time equals the total dedicated CPU time in case of single thread execution, and corresponds to a fraction of it in case of algorithm supporting multi thread execution. In our experiments only RFI implemented parallel execution, and it was configured to use 10 CPUs, thus all the reported times are to be considered as time elapsed while employing ten times more computational resources than the other algorithms.\\
To ensure consistency in the experimental conditions, all analysis were run on the same computing platform: a multi-node cluster with 64 CPUs AMD Opteron(tm) 2400 MHz and 256GB RAM for each node.
 
\subsection{Software}
\label{sec:software}
Data handling, editing and preparation, summary of results and plots were all performed using the open-source environment for statistical programming R \cite{r_core_team_r:_2014}. All imputation methods except Beagle and FILLIN were implemented in R: MNI directly in base R; KNNI using the function \emph{knncatimpute()} from the Scrime package \cite{schwender_scrime:_2013}; SVDI with the R package ``bcv'' \cite{perry_bcv:_2009}; RFI using the ``MissForest'' \cite{stekhoven_missforestnon-parametric_2012} R package (with parameters: ntree=100, maxiter=10, parallelize='variables'). The ``Beagle'' localized haplotype clustering imputation method was implemented with the Beagle software version 3.3.2 \cite{browning_rapid_2007}. The FILLIN algorithm is implemented using the Tassel CLI plugin \emph{-FILLINFindHaplotypesPlugin} followed by \emph{-FILLINImputationPlugin}. The latter allowed accuracy measurements through \emph{-accuracy} and \emph{-propSitesMask} options.\\
The Bowtie 2 tool \cite{langmead_fast_2012} was used to query the consensus sequence of each alfalfa tag pair containing a SNP against the \emph{M. truncatula} reference genome Version 4.0 using the \emph{-verysensitivelocal} preset.
\textcolor{red}{TO BE WRITTEN: what about Rice alignment, what sw was used? How many marker did align? We need to report stats for alfalfa, too? (about half of the marker aligned).}


