\section{Conclusions}
\label{sec:conclusions}
Imputation of missing genotypes can be an effective technique also for GBS data.
The most accurate imputation methods resulted in a total amount of wrongly imputed genotypes near to zero in rice and slightly over 10\% in alfalfa. 
The proportion of imputation errors, however, varied dramatically among genotype classes, approaching very high levels with the worst methods on the minor homozygous class.
The structure of the genome and the maturity of the reference assembly play a role in the imputation efficiency, as indicated by the greater efficiency obtained in rice compared with alfalfa. While Beagle was preferable for rice, general data imputation methods performed significantly better in the absence of a reference genome. In particular, in alfalfa RFI and KNNI showed the lowest imputation errors in all classes, with KNNI to be preferred for computational efficiency. The alignment of markers to closely related species can help, but comes at the price of discarding the non-aligning markers. Moreover, performances indicate that general methods are still the best option.\\
The imputation accuracy tended to be relatively robust over increasing missing rates. However, KNNI showed a somewhat lower accuracy on high missingness scenarios, probably as a consequence of the ``curse of dimensionality''. In terms of computation requirements, all methods proved to be tractable over all problem complexities with a standard bioinformatics-lab computation infrastructure, except RFI (whose computational requirements increased exponentially with problem size and took up to 40 days with the complete rice dataset and maximum missing rate thresholds).\\
Exploring ways to improve imputation accuracy in GBS data, for instance by combining predictive methods as in boosting or model ensembling, while at the same time increasing algorithm efficiency to keep computation times reasonably low, is an important research area, and can help making low-cost genotyping-by-sequencing a very cost-effective technique for genomic applications in species with and without a reference genome.

