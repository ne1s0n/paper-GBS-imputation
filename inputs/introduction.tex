\section{Introduction}
\label{intro}

Imputation of missing alleles and genotypes is a preliminary step for
a wide range of genetic analyses. In fact most models and software for 
population genetics, genomic selection (GS) and genome-wide association 
studies (GWAS) can not handle missing data and require complete datasets (e.g. \cite{hayes2009invited,aulchenko2007genabel,endelman2011rrblup,perez2014genome}). 
SNP-array represent one of the leading genotyping technologies and produce 
datasets that, after low call-rate filtering, 
usually still contain a small proportion of uncalled genotypes (e.g., $<$5\%) 
randomly distributed along the genome. Genotyping-by-sequencing (GBS) is a 
relatively recent technique which is considered a viable alternative to SNP array-based genotyping to produce SNP genotype data~\cite{elshire_robust_2011}.
GBS is platform-independent and is conveniently 
used in species that lack commercial SNP-chips or even lack a reference 
genome sequence. GBS data typically present a much larger proportion of 
sporadic missing genotypes: e.g., $\sim$52\% and $\sim 58\%$ average in two maize experimental populations \cite{crossa_genomic_2013},
or ``up to 80\% missing data per marker'' in wheat \cite{poland_genomic_2012}. This 
is due to both intrinsic properties of the technology
and to its application to species without a mature genome assembly \cite{glaubitz_tassel-gbs:_2014}.\\
There are several methods that are routinely used for the imputation of 
missing genotypes. Some rely solely on genotypic information, some make 
use also of genealogies, most of them are based on reconstructing haplotypes 
to be used in predictive models \cite{nicolazzi_software_2015}. Most methods for 
genotype imputation have been applied to SNP array data, and typically 
yield very high imputation accuracy. For instance, $>$95\% correctly imputed 
genotypes were reported in maize \cite{hickey_factors_2012} and cattle \cite{vanraden_genomic_2011}. 
The same imputation methods can also be applied to GBS data (see \cite{huang_efficient_2014} and \cite{swarts_novel_2014} for applications in rice and maize). When a reference 
genome is available, SNP loci can be aligned against its sequence and 
are therefore ordered, thus allowing the exploitation of local linkage disequilibrium (LD).
However, GBS data may be generated 
also for species lacking a reference genome assembly. In this case the 
GBS output is a series of ``floating'' loci not linked to any chromosome:
alignment is not possible and SNPs are therefore considered unordered. Unordered 
data add additional complexity to genotype imputation. Exploitation of linkage 
disequilibrium and haplotype reconstruction 
are less straightforward, consecutive loci may lie on different chromosomes
or scaffolds, and overall data are less homogeneous.
Rutkoski et al.~\cite{rutkoski_imputation_2013}
considered the application of general data imputation methods 
to wheat, maize and barley populations, obtaining best case scenarios accuracies 
as high as 0.84-0.94 (measured as $R^{2}$ between true and imputed genotypes).
Imputation of missing GBS data without alignment to the reference genome has
been reported in alfalfa (``Medicago sativa'', \cite{Rocher_validation_2015})
and red raspberry (``Rubus idaeus'', \cite{ward_saturated_2013}). These studies
however do not focus on measuring imputation efficiency and do not compare
alternative imputation models.\\
	%un po' una ripetizione. Togliere?
	Compared to SNP-array data, GBS is more challenging: many more 
	missing genotypes, high variability between runs, intrinsically noisy data, 
	reads from different loci that can overlap. For all these reasons, imputation 
	of missing genotypes with GBS data is still a relatively immature technique, 
	not yet amenable to produce highly standardised and repeatable results. 
There is therefore scientific and practical interest in gaining further 
insights into how genotype imputation with GBS data works, and in 
developing the best methods and strategies to accurately and efficiently 
impute such missing data. This would be relevant not only for the scientific 
community but also for the breeding industry, because of its direct 
consequences on genomic applications.\\
In this paper, we imputed missing genotypes from GBS data in two agronomically
important crop species:
the cereal crop rice (\emph{Oryza sativa L.}) whose reference genome has been 
assembled in 2005 \cite{international_rice_genome_sequencing_project_map-based_2005},
and updated in 2013 \cite{kawahara_improvement_2013},
and
the forage crop alfalfa (\emph{Medicago sativa L.}), for 
which a reference genome is not available yet. When necessary, alfalfa genotypes
were mapped on the close relative \emph{Medicago truncatula L.}, a model species
for legumes, whose genome is available \cite{young_medicago_2011}.
The species were chosen 
as representative of very different scenarios, alfalfa being autotetraploid
with high heterozygosity and rice being diploid with essentially 100\% homozygozity.
The presented study thus encompasses a wide span of real application cases.\\
We compared the imputation efficiency of four general imputation methods (Mean
Value Imputation, K-Nearest Neighbor Imputation, Singular Value Decomposition
Imputation and Random Forest Imputation) with the performance of two methods specific 
for genotype imputation (localized haplotype clustering imputation, 
implemented in the software package Beagle, \textcolor{red}{and haplotype
reconstruction around the recombination site}, implemented in the FILLIN algorithm
as part of Tassel suite). The four general algorithms were chosen as well known 
imputation strategies implemented in several freely available software libraries
and frameworks. The two genotype-specific methods represent the state of the art 
for genotypes lacking pedigree information.\\
The relative efficiency of the different imputation methods was assessed in terms of 
accuracy and computation time. Accuracy was measured as fraction of 
correctly imputed genotypes, and within genotype classes 
(major/minor homozigotes, and heterozigotes, when present).

