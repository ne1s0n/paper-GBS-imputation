\section{Introduction}
\label{intro}

Imputation of missing alleles and genotypes is usually a preliminary step for
a wide range of genetic analyses: in fact, most models and software for 
population genetics, genomic selection (GS) and genome-wide association 
studies (GWAS) can not handle missing data and require complete datasets (e.g. \cite{hayes2009invited}). 
SNP-chips represent one of the leading genotyping technologies, and produce 
datasets that, after markers and individuals with low call-rate are edited out, 
usually still contain a small proportion of uncalled genotypes (e.g., $<$5\%) 
randomly distributed along the genome. Genotyping-by-sequencing (GBS) is a 
relatively recent technique which is considered a viable alternative to SNP array-based genotyping to produce SNP genotype data~\cite{elshire_robust_2011}.
GBS is platform-independent and is conveniently 
used in species that lack commercial SNP-chips or even lack a reference 
genome sequence. GBS data typically present a much larger proportion of 
sporadic missing genotypes: e.g., $\sim$52\% and $\sim 58\%$ average in two maize experimental populations \cite{crossa_genomic_2013},
or ``up to 80\% missing data per marker'' in wheat \cite{poland_genomic_2012}. This 
is due to both intrinsic properties of the technology - e.g., low coverage - 
and to its application to species without a mature genome assembly \cite{glaubitz_tassel-gbs:_2014}.\\
There are several methods that are routinely used for the imputation of 
missing genotypes. Some rely solely on genotypic information, some make 
use also of genealogies, most of them are based on reconstructing haplotypes 
to be used in predictive models \cite{nicolazzi_software_2015}. Most methods for 
genotype imputation have been applied to SNP array data, and typically 
yield very high imputation accuracy: for instance, $>$95\% correctly imputed 
genotypes were reported in maize \cite{hickey_factors_2012} and cattle \cite{vanraden_genomic_2011}. 
Such methods can be applied to GBS data, too (see for instance 
\cite{huang_efficient_2014} and \cite{swarts_novel_2014} 
for applications in rice and maize). When a reference 
genome is available, SNP loci can be aligned against its sequence and 
are therefore ordered. However, GBS data may be generated 
also for species lacking a reference genome assembly. In this case the 
GBS output is a series of ``floating'' loci not linked to any chromosome:
alignment is not possible and SNPs are therefore unordered. Unordered 
data add additional complexity to the task of imputing missing genotypes: 
exploitation of linkage disequilibrium (LD) and haplotype reconstruction 
are less straightforward, consecutive loci may lie on different chromosomes
or scaffolds, and overall data are less homogeneous. Rutkoski et al.~\cite{rutkoski_imputation_2013}
considered the application of general data imputation methods 
to wheat, maize and barley populations, obtaining best case scenarios accuracies 
as high as 0.84-0.94 (measured as $R^{2}$ between true and imputed genotypes). Imputation of missing GBS data without alignment to the reference genome has been reported in alfalfa (``Medicago sativa'', \cite{Rocher_validation_2015}) and red raspberry (``Rubus idaeus'', \cite{ward_2013_saturated}).\\
Compared to SNP-chip data, GBS data are more challenging: many more 
missing genotypes, high variability between runs, intrinsically noisy data, 
reads from different loci that can overlap. For all these reasons, imputation 
of missing genotypes with GBS data is still a relatively immature technique, 
not yet amenable to produce highly standardised and repeatable results. 
There is therefore scientific and practical interest in gaining further 
insights into how genotype imputation with GBS data works, and in 
developing the best methods and strategies to accurately and efficiently 
impute such missing data. This would be relevant not only for the scientific 
community but also for the breeding industry, because of its direct 
consequences on genomic applications.\\
In this paper, we imputed missing genotypes from GBS data in two agronomically
important crop species: the forage crop alfalfa (\emph{Medicago sativa}), for 
which a reference genome is not available yet, and the 
cereal crop rice (\emph{Oryza sativa}) whose reference genome has been 
assembled in 2005 \cite{international_rice_genome_sequencing_project_map-based_2005}, and
was recently updated \cite{kawahara_improvement_2013}. We compared the imputation 
efficiency of four general data-imputation methods (Mean Value Imputation, 
K-Nearest Neighbour Imputation, Singular Value Decomposition Imputation 
and Random Forest Imputation) with the performance of two methods specific 
for genotype imputation (localized haplotype clustering imputation, 
implemented in the software package Beagle, %and haplotype reconstruction around the recombination site
and the FILLIN algorithm implemented in the programming suite Tassel). The relative efficiency of 
the different imputation methods was assessed in terms of  
accuracy and computation time. Accuracy was measured in terms
of gross fraction of correctly imputed genotypes, and within
genotype classes (major/minor homozites and heterozigotes, when present).
